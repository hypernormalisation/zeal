\documentclass[letterpaper,11pt]{article}
\usepackage{tabularx}
\usepackage{amsmath}
\usepackage{graphicx}
\usepackage[margin=1in,letterpaper]{geometry}
\usepackage[final]{hyperref}
\usepackage{lineno}
\usepackage{siunitx}
\usepackage{xspace}
\usepackage{multirow}
\usepackage{enumitem}% http://ctan.org/pkg/enumitem
%%%%%%%%%%%%%%%%%%%%%%%%%%%%%%%%%%%%%%%%%%%%%%%%%%%%%%%%%%%%%%%%%%%%%%%%

\setlist[description]{labelindent=25pt,style=multiline,leftmargin=2.5cm}
\linenumbers
\newcommand{\ws}{\ensuremath{s_\mathrm{w}} \xspace}
\newcommand{\wdam}{\ensuremath{d_\mathrm{w}} \xspace}
\newcommand{\psoc}{\ensuremath{p_\soc} \xspace}
\newcommand{\soc}{\ensuremath{\mathrm{SoC}} \xspace}
\newcommand{\sob}{\ensuremath{\mathrm{SoB}} \xspace}
\newcommand{\dave}{\ensuremath{d_{\mathrm{ave}}} \xspace}
\newcommand{\dswing}{\ensuremath{d_{\mathrm{swing}}} \xspace}
\newcommand{\dphys}{\ensuremath{d_{\mathrm{phys}}} \xspace}
\newcommand{\dholy}{\ensuremath{d_{\mathrm{holy}}} \xspace}
\newcommand{\armor}{\ensuremath{a}\xspace}
\newcommand{\lvl}{\ensuremath{l}\xspace}
\newcommand{\pwf}{\ensuremath{p_{\mathrm{WF}}} \xspace}
\newcommand{\dr}{\ensuremath{DR(\%)} \xspace}
\hypersetup{
	colorlinks=true,       % false: boxed links; true: colored links
	linkcolor=blue,        % color of internal links
	citecolor=blue,        % color of links to bibliography
	filecolor=magenta,     % color of file links
	urlcolor=blue         
}
%\setlength{\parindent}{0pt} 

%%%%%%%%%%%%%%%%%%%%%%%%%%%%%%%%%%%%%%%%%%%%%%%%%%%%%%%%%%%%%%%%%%%%%%%%

\begin{document}
	
	\title{The Mathematics of Ret Paladin in BCC}
	\author{Swedge}
	\date{\today}
	\maketitle
	
	\begin{abstract}
		This document details considerations relevant to a Burning Crusade Classic (BCC) retribution paladin's rotation and calculations for relative damage outputs of various rotations, in an attempt to help players understand paladin rotations and to ensure the rotations we use are the formally optimal ones.
	\end{abstract}
	
	%%%%%%%%%%%%%%%%%%%%%%%%%%%%%%%%%%%%%%%%%%%%%%%%%%%%%%%%%%%%%%%%%%%%%%%%
	\section{Introduction}
	
	Retribution paladin in BCC relies heavily on ``seal twisting'' to increase the class's damage output.
	This technique of swinging with two active seals was an artefact of ``spell batching'' in retail BC, where as an optimisation the server processed batches of actions that would be evaluated together.
	
	In BCC, the batch size was greatly reduced, but some seals were changed to persist for a very short time after switching to another seal.
	This allows the paladin to change seals very shortly before a swing, and to have both seals active. For reasons we will cover when looking at the paladin's abilities, this results in some multiplicative damage effects over strictly additive ones, significantly increasing the paladin's damage output.
	
	In this document, we will first examine the ret paladin's abilities and the buffs we commonly utilise.
	We will then define expected damage outputs as a relative fraction of weapon damage, starting with simple situations and later taking into account more complex considerations.
	This allows us to somewhat ignore what gear the ret is using, with some notable exceptions.
	
	This is, of course, no substitute for a sim - but it can help theorycrafters to some extent quantify the expected damage output and value of different rotations, and perhaps to find new rotations that had been previously considered unviable.
	
	%%%%%%%%%%%%%%%%%%%%%%%%%%%%%%%%%%%%%%%%%%%%%%%%%%%%%%%%%%%%%%%%%%%%%%%%
	\section{Assumptions}
	We make the following assumptions that generally hold in a PvE raiding scenario:
	\begin{itemize}
		\item the ret is lvl 70
		\item the ret is in a raid group with an enhancement shaman casting Windfury Totem with $100~\%$ uptime
		\item the ret is attacking a boss enemy at lvl 73
		\item the ret's weapon skill is maxed out at 350
		\item the ret is hit-capped, and has sufficient hit rating to remove misses from the attack table
		\item the ret is hitting the target from behind, so cannot be parried
		\item the ret is \emph{not} dodge-capped at $6.5\%$, meaning there is some finite chance for their attacks, seals, and some abilities to be dodged 
	\end{itemize}
	
	%%%%%%%%%%%%%%%%%%%%%%%%%%%%%%%%%%%%%%%%%%%%%%%%%%%%%%%%%%%%%%%%%%%%%%%%
	\section{The BCC Combat System}
	In this section we detail certain facets of the BCC combat system and character statistics relevant to a ret paladin's performance.
	
	\subsection{The Attack Table}
	Basic attacks in TBC are determined as a single random roll on the server.
	This means that every possible outcome for the attack is weighed together in one roll (as opposed to e.g. a multiple roll system that first evaluates the probability of an attack to
	hit or miss, and then subsequently makes a second independent roll upon a hit to evaluate critical strike chance, glancing chance etc).
	
	Each outcome of the single roll is assigned a certain precedence, such that higher precedence outcomes have the capability under certain conditions to ``push'' lower precedence outcomes off of the attack table.
	These outcomes are:
	\begin{description}
		\item[miss] the attack misses the target and is negated.
		\item[dodge] the target dodges and the attack is negated.
		\item[parry] the target parries the attack, negating the attack and hasting their autoattack timer. Parries can only occur when attacking the target from the front.
		\item[glance] when the target is three levels or more higher than the attacker, there is a chance for the attack to be a glancing blow, which deals reduced damage. The damage reduction on the attack is rolled uniformly between $15 - 35 \%$, for an average reduction of $25\%$. When the attacker is three levels lower than the target, the chance for a glancing blow to occur is $24\%$. % should move this to its own section and give formulae
		\item[block] the target blocks the attack, reducing the damage taken from the attack in accordance with the incoming damage and the target's \emph{Block Value} statistic.
		\item[crit] the attack is a critical hit, dealing a base of $200\%$ the regular damage. Critical strike damage can be modified by gear and talents.
		\item[hit] the attack is a hit, dealing regular damage.
	\end{description}
	In practical PvE scenarios, the ret is hitting the target from behind.
	As such, the block and parry outcomes are removed from the attack table.
	The miss and dodge outcomes can be reduced or eliminated entirely through appropriate gear and talents.
	In our scenario, the target enemy is three levels higher than the ret and therefore glancing blows must be accounted for in the attack table.
	Glancing blow chance cannot be reduced.
	
	\subsection{The Global Cooldown (GCD)}
	The Global Cooldown or GCD is a period following an ability cast (and not a swing or seal hit) where no further casts are possible.
	The base duration of the GCD is 1.5s, but it can be reduced by Spell Haste.
	Ret paladin's commonly acquire Spell Haste through the Shaman's Bloodlust ability and the Leatherworker's Drums of Battle, which together reduce the GCD to 1.1s.
	
	The GCD is of particular importance to a ret paladin's ability to seal twist.
	In order to twist e.g. a SoC into a SoB, the SoC ability (which is typically not the seal at the start of the swing, favouring SoB under most scenarios) must be cast more than a GCD before the swing timer completes such that the paladin can cast the secondary seal before the swing completes and the attack is made.
	
	We also note that the Crusader Strike ability is not a cast and does not benefit from Spell Haste, meaning the GCD following its use is \emph{always} 1.5s.
	
	\subsection{Internal Cooldowns (ICD)}
	Internal Cooldowns refer to the time period where a given spell or ability cannot proc after proccing.
	
	The relevant ICDs to a ret paladin are:
	\begin{itemize}
		\item Seal of Command, which has a 1s ICD to prevent it from proccing on both an autoattack and Windfury Attack on the same swing
		\item Windfury Attack itself, which has a very short ICD to prevent it from proccing on both an autoattack and a Seal of Command attack.
	\end{itemize}

	\subsection{Armor}
	Armor is a stat that reduces the amount of physical damage received by a target.
	All targets have an armor value, and BCC bosses typically have either 6200 or 7700 armor.
	The target's level also influences the amount of damage reduction that a given level of armor will provide.
	The damage reduction formula for targets over level 60 is given as
	\begin{equation}
		\dr = \frac{100(\armor + 1)}{467.5 \lvl -22167.5}
	\end{equation}
	where \armor is the target's armor value, and \lvl is the target's level.
	A significant proportion of a paladin's damage is physical, and so the target's armor level must be considered when evaluating rotations.
	
	\subsection{Spell Resistance}
	In BCC, spells have a chance to be resisted either fully, such that the spell is nullified, and/or partially in the case of many damaging spells.
	Partial resists result in the spell's damage being resisted by either $25\%$, $50\%$, or $75\%$.
	The chance to be resisted is dependent on the level difference between the caster and the target, the caster's Spell Penetration stat, and the target's resistance to the relevant school of magic.
	
	Some of the ret paladin's offensive seals are partially implemented as spells that do holy damage.
	They cannot be fully resisted, but they can be partially resisted.
	Holy is not a school of magic for which spell resistance can be acquired.
	Therefore the chance for a seal to be partially resisted is therefore entirely dictated by the level difference between the caster and target.
	
	Against raid bosses, the level difference is always equal to 3.
	In this case, the probabilities for a seal to be partially resisted are given in Table \ref{tab:glancing}.
	\begin{table}[htb]
		\centering
			\begin{tabular}{r | l}
				Resistance (\%) & Probability \\
				\hline \hline
				0 & 0.82 \\
				25 & 0.13 \\
				50 & 0.04 \\
				75 & 0.01 \\
				100 & 0 \\
				\hline
			\end{tabular}
	    	\caption{Probabilities for spell partial resists against a Boss target}		
			\label{tab:glancing}
	\end{table}
	Integrating these values gives an average damage resistance against a ret paladin's seals of $6\%$.

	\subsection{Hit Rating and Hit Chance}
	Against level 73 (or Boss level) targets, wielders of only a single weapon need a total of $9\%$ Hit Chance to never miss a target.
	Hit rating is found on gear, and increases Hit Chance.
	At level 70, 15.8 Hit Rating increases your Hit Chance by $1\%$. 
	Paladins have access to the talent \emph{Precision}, which increases the Hit Chance and Spell Hit Chance by $3~\%$.
	The Balance specialisation of the Druid class has access to the talent Improved Faerie Fire, which increases the targets chance to be hit by $3~\%$ when affected by the Faerie Fire debuff.
	In most raiding scenarios, the ret paladin must therefore acquire at least 48 Hit Rating from their gear to attain an additional $3\%$ Hit Chance in order to fully remove the chance to miss from the attack table.
	This is almost always very trivial, and so we do not need to consider miss chance in our considerations.
		
	\subsection{Expertise Rating and Expertise}
	When melee attacking a target 3 levels higher, a player has a base $6.5\%$ chance to be dodged.
	As parries can only occur when attacking the target from the front, and we attack the target from behind, parries are usually not present in the attack table.

	Each point of Expertise reduces the target's chance to dodge or parry an attack by $0.25\%$.
	Expertise Rating is found on gear.
	At level 70, 3.9 points of Expertise Rating are required to give one point of Expertise.
	As such, a player requires 26 Expertise to totally remove dodge chance from the attack table.
	
	Ret paladins do not typically take any talents that provide Expertise.
	As such, most races of paladin must acquire 102 points of Expertise Rating from their gear in order to reach 26 Expertise.
	Humans have a racial passive that provides them with 5 Expertise when using a sword or mace.
	Human rets when using the appropriate weapon only require 82 points of Expertise Rating in order to reach 26 Expertise.
	
	Because Expertise Rating is relatively scarce on gear, most paladins retain some chance to be dodged even when wearing excellent or even optimal gear.
	For instance, a non-human ret can only acquire a maximum of 95 Expertise Rating in phase two of BCC, and therefore retains a $0.5\%$ chance to be dodged.
	Due to Expertise typically being available on gear only in large increments, it is not optimal for human rets in phase two to run a gear setup that fully eliminates dodge chance.
	
	As we will cover when looking at the ret paladin's abilities later in this document, Seal spells also have a chance to be dodged.
	Because of the chained nature of a paladin's damage rolls, ret paladin's benefit more from Expertise than any other melee class in BCC.
	
	\subsection{Attack Power (AP)}
	Attack Power increases base melee damage per second by 1 point for every 14 Attack Power.
	When considering rotations, this stat is of note because Windfury Attacks from an enhancement Shaman's Windfury Totem have a bonus to Attack Power.
	Therefore rotations that favour twists and SoC swings will benefit \emph{very slightly} from the extra AP on those attacks because SoC can proc Windfury Attack.
	
	\subsection{Critical Strike Chance}
	Critical strike chance is the chance for an attack to critical strike.
	It is relevant to evaluating rotations because of an interaction between the Libram of Avengement item and the Judgement of Blood ability (detailed later).
		
	\subsection{Critical Strike Damage}
	Critical Strike Damage accounts for any improvements to the base $200\%$ critical strike damage.
	Ret paladins without exception should use the Relentless Earthstorm Diamond as their helmet's meta gem, which provides $3~\%$ increased critical damage.
	As such, a typical critical strike for a ret paladin will do $206\%$ of the base damage.
	
	\subsection{Spellpower}
	A player's Spellpower increases the damage of their spells, by a multiple of their total Spellpower multiplied by the spell's Spell Damage Coefficient.
	Spellpower can affect all schools of magic or individual schools.
	Ret paladin's optimally do not use any gear that includes spellpower, however one of the paladin's judgement debuffs provides a flat amount of spellpower to the Holy school.
	Some of the important abilities in ret paladin rotations have Spell Damage Coefficients, so we must include this in our calculations.
	The spell coefficients for the spells relevant to a ret paladin's rotation are provided in Table \ref{tab:spellcoefficients}.
	
	\begin{table}[htb]
		\centering
		\begin{tabular}{r | l}
			Spell & Coefficient \\
			\hline \hline
			Consecration & 0.9524 \\
			Exorcism & 0.4286 \\
			Hammer of Wrath & 0.4286 \\
			Judgement of Blood & 0.43 \\
			Seal of Command & 0.2 \\
			\hline
		\end{tabular}
		\caption{Spellpower Coefficients relevant to a ret paladin.}		
		\label{tab:spellcoefficients}
	\end{table}
	
	%%%%%%%%%%%%%%%%%%%%%%%%%%%%%%%%%%%%%%%%%%%%%%%%%%%%%%%%%%%%%%%%%%%%%%%%
	\section{Ret Abilities}
	We will now detail the most important abilities to a ret's rotation.

	\subsection{Autoattacks and Windfury Attacks}
	The plurality of a ret's damage output comes from autoattack damage and Windfury Attacks.
	A shaman's Windfury Totem has a short internal cooldown, that prevents a melee attack and a SoC attack (detailed later) from proccing on the same swing.
	As our hypothetical ret is hit-capped, misses are removed from the attack table, but they are
	not dodge-capped, so dodges will still be present in accordance with the ret's expertise stat.
	Autoattacks and Windfury Attacks can both, of course, be dodged.
	
	Windfury Attacks get a bonus to AP, so they are more damaging than regular attacks by a constant damage factor.
	This means that as the ret's total amount of AP goes up, the difference in the average damage between a regular attack and a windfury attack goes down.
	
	Windfury attacks caused by the max rank Windfury Totem get a bonus to attack power of 445AP, which is usually increased by $15~\%$ by taking one rank in Improved Weapon Totems talent in the Enhancement tree for a total of 511.75AP.
	This adds approximately 41.3dps to the attack, meaning that the damage added to the attack is dependent on the weapon's base speed.
	For e.g. $\ws = 3.6~\mathrm{s}$ the attack's damage is increased by approximately 131 damage.
	The formula to calculate this extra damage is given by:
	
	\begin{equation}
		d = \frac{511.75 \ws}{14}
	\end{equation}
	
	\subsection{Seals and Judgements}
	Seals are 30s duration buffs the paladin casts on themselves.
	The effect of a seal can be expended by using the Judgement ability to produce a damaging action or to cast a debuff on the target.
	Judgement is notably off the GCD, meaning it can be cast while the GCD is ticking.
	Retribution paladins do a significant portion of their damage as holy damage through offensive seals that enhance their auto-attacks.

		
	\subsubsection{Seal of Command (SoC)}
	This ability gives the paladin a chance to do $70\%$ of their weapon damage as holy on their attack.
	The chance is determined by the base weapon speed of the paladin's weapon, and is normalised at 7 procs per minute (PPM).
	The proc chance of SoC is therefore given by:
	
	\begin{equation}
		\psoc = \frac{7 \ws}{60}
	\end{equation}
	where \ws is the base swing speed of the paladin's weapon in seconds.
	This is the primary reason that ret paladins favour slow weapons - under no melee haste and neglecting Windfury hits, a fast-swinging weapon procs SoC on average 7 times for low damage due to the PPM normalisation, while a slow-swinging weapon procs SoC 7 times on average per minute for high damage.
	The proc chance is unchanged by haste effects, meaning that one can proc SoC well over 7 times per minute while under high haste.
	There is no limit on how many times SoC can proc in any given minute.
	
	\begin{itemize}
		\item SoC has an internal cooldown (ICD) of 1s, meaning that it cannot be procced on e.g. both melee hits in a windfury attack (but it can proc on a windfury attack if it \emph{does not} proc on the initial melee hit).
		\item SoC \emph{lingers}, meaning it persists for 400~ms after the paladin casts another seal. As such, it can be used as the first seal in a twist attempt.
		\item SoC is implemented as an attack that does Holy damage, and is subject to the basic attack table and to partial resists.
		\item It has a spell damage coefficient of 0.2 (given in \tabref{tab:spellcoefficients}).		
		\item SoC can proc Seal of Blood (See \secref{sec:sob}).		
	\end{itemize}
	
	\subsubsection{Judgement of Command}
	When the ret has SoC active, the Judgement ability will deal an amount of holy damage to the enemy, which is increased against incapacitated targets.
	Because SoC is the best seal to start a twist from, its judgement is never used in PvE content.
	As such we will not go into detail on its damage, or any related libram effects that improve SoC's judgement.
	
	\subsubsection{Seal of Blood (SoB)}
	\label{sec:sob}
 	This seal gives the paladin a $100 \%$ chance to do $35 \%$ of their weapon damage as holy on their attack, at the cost of the paladin losing health equal to $10 \%$ of the damage inflicted.
	\begin{itemize}
		\item SoB has no ICD, meaning it can proc on both hits of a swing under Windfury Attack.
		\item SoB does not linger, meaning that it can only be twisted into, and not as the first seal in a twist.
		\item SoB is implemented as an attack that deals holy damage, and is therefore subject to the attack table and to partial resists.
		\item SoB cannot proc SoC or Windfury Attack.
		\item it can be procced by SoC, resulting in three instances of SoB damage in a windfury twist.
	\end{itemize}
 	
	\subsubsection{Judgement of Blood}
	When the ret has SoB active, the Judgement ability will deal 295 to 325 Holy damage at the cost of health equal to $33 \%$ of the damage caused.
	The best Libram for rets in TBC is available starting in phase 1, the Libram of Avengement (see \figref{fig:loa}).
	\begin{figure}[ht] 
		\centering \includegraphics[width=0.44\columnwidth]{figs/libram_of_avengement.png}
		\caption{The Libram of Avengement}
		\label{fig:loa}
	\end{figure}
	This libram causes several judgements, including Judgement of Blood, to increase the ret paladin's critical strike rating by 53 for 5 seconds.
	As such, any rotations that provide the ret time to judge blood will affect the expected damage of the ret's attacks and seals for a subsequent 5 seconds, and critical strike chance must be included in any rigorous damage projections.
	
	\subsection{Crusader Strike (CS)}
	CS is an instant cast strike that deals $110 \%$ weapon damage to the target, and refreshes all judgements on the target.
	It has a 6 second cooldown.
	
	\todo{figure out exactly how the Crusader Strike weapon damage normalisation at 3.3s actually works and how it affects the outgoing damage of the ability.}
	
	\subsection{Consecrate}
	Consecrate is an instant-cast spell that marks an area underneath the paladin, that at maximum rank deals 512 Holy damage to all enemies who are in or enter the area over 8s.

	\todo{fill out info on filler spells.}
	
	\subsection{Exorcism}
	Exorcism is an instant-cast, single-target spell that at maximum rank causes 626 to 698 Holy damage, that can only be cast on Undead or Demon targets.
	
	
	\subsection{Judgement of the Crusader}
	One of the paladin's seals is the Seal of the Crusader.
	The effects of the seal are useless in PvE (outside of levelling weapon skill due to the haste on attacks) but when a properly specced ret paladin judges this seal on a target, a powerful debuff is cast.
	By default, the judgement increases the Holy damage taken by the target (by 219 at max rank).
	The ret paladin also takes 3/3 of the talent Improved Seal of the Crusader, which increases the targets chance to be critically hit \emph{by all attacks} (meaning all attacks and spells) by $3~\%$.	
	
	\subsection{Sanctity Aura}
	Paladins have access to a suite of passive passive party buffs called ``auras''.
	A paladin can only have one aura active at a time, but a party with multiple paladins can benefit simultaneously from multiple auras.
	Aura effects of the same kind do not stack.
	
	Without exception, ret paladin's should utilise Sanctity Aura, which gives all holy damage done by the party a $10~\%$ increase.
	This factors into the relative weight of a ret paladin's physical and holy damage, and should be considered in considerations of rotations.
	The talent 2/2 Improved Sanctity Aura is also taken by ret paladins, which improves the amount of damage caused by targets subject to Sanctity Aura by $2~\%$.
	
	We note that the holy damage increase from Sanctity aura does not interact with the holy damage increase from Judgement of the Crusader.
	
	%%%%%%%%%%%%%%%%%%%%%%%%%%%%%%%%%%%%%%%%%%%%%%%%%%%%%%%%%%%%%%%%%%%%%%%%

	input{symbols}

\section{Measuring and Projecting Damage}
Now that we have detailed BCC's combat system and its facets relevant to a ret paladin's damage output, we may now turn to the task of quantifying the projected damage of various actions a ret paladin can take.
It is of interest to be able to compare projected damage outputs of various individual actions and rotations without having to simulate entire gear sets.
Unfortunately, many aspects of a paladin's gear directly affect the relative effectiveness of different actions and rotations.
These effects are often small, but can be significant.
For example:
\begin{itemize}
	\item the paladin's Expertise affects the chance for their attacks to be dodged. When the dodge chance is lower, swing actions that involve many segments of damage being chained together increase in relative value to ``simpler'' swings, as a dodge breaking the damage chain is less likely to occur.
	\item the Libram of Avengement crit buff is \emph{relatively} more effective in gear sets that have low amounts of Critical Strike Chance, and therefore rotations that provide the ability to use judgement on cooldown benefit rise in \emph{relative} value (though the effects are almost certainly very small)
\end{itemize}
It is desirable to reduce gear sets to the lowest amount of input variables possible such that these effects can be accounted for.

\subsection{Damage Projections and Correlation}
Much of a ret paladin's damage output is already (to some extent) normalised to weapon damage.	
Seal of Blood, Seal of Command, and Crusader Strike are all typically expressed in terms of a percentage of the paladin's weapon damage.
The average weapon damage of a paladin's basic autoattack is therefore a useful metric to start with.
We first define the average weapon damage as the damage that occurs when a weapon rolls a median value on its damage range, and is a regular hit (i.e. not a crit or a glancing blow).
\begin{equation}
	\dave = \fimpsanc \ftwoh \fcrusade \left( \bar{\wdam} + \frac{x\ws}{14} \right)
\end{equation}
where \fimpsanc is the scale factor from Improved Sanctity Aura, \ftwoh is the scale factor from the Two-handed Weapon Specialisation talent, \fcrusade is the scale factor from the Crusade talent (which is simply 1.0 if not fighting a humanoid, undead, demon, or elemental), $\bar{\wdam}$ is the median damage on the weapon's damage range, $x$ is the paladin's attack power, and \ws is the paladin's weapon speed.

We note that when considering \emph{relative} damages of actions or rotations, global multipliers to outgoing damage like those from the Improved Sanctity Aura or Two-handed Weapon Specialisation talents are not important, because they will always cancel out in the ratio between the two choices.

\subsubsection{Attacks}
Let us first consider the damage \dauto of so-called ``white hits'', meaning a simple autoattack.
The \dave figure is useful because it allows for all results of an attack table roll to be expressed in some multiple of \dave:
\begin{equation}
	\dauto = \dave \sum_n p_{\mathrm{n}} f_{\mathrm{n}}
\end{equation}
where the sum is over the $n$ possible outcomes for the attack roll that are left on the table (e.g. crit, dodge \ldots), with each scenario having a probability to occur $p_{\mathrm{n}}$, and a damage scaling factor for its outcome of $f_{\mathrm{n}}$ (e.g. $\fcrit = 2.06$ with an activated Relentless Earthstorm Diamond meta-gem).
Giving these outcomes explicitly, we arrive at the expression:
\begin{equation}
	\dauto = \dave (\pdodge \fdodge + \pglance \fglance + \pcrit \fcrit + \phit \fhit)
\end{equation}
where \pdodge and \fdodge are the probability and damage scaling of a dodge, and the other outcomes correspond to glances, crits, and regular hits. 

This expression can be simplified first by giving $\fdodge = 0$ (as dodges are negated entirely), \fhit as simply 1 (as normal hits to standard damage), and then describing \phit as the difference of 1 and the probability of all other attack outcomes (given hit expands to take up the remainder of the attack table), or
\begin{equation}
	\phit = 1 - \pglance - \pdodge - \pcrit
\end{equation}
The expression for the projected autoattack damage then becomes:
\begin{equation}
	\dauto = \dave (\pglance \fglance + \pcrit \fcrit + (1 - \pdodge - \pglance - \pcrit))
\end{equation}

The above expression projects the expected average damage from a simple autoattack.
We note, however, that many components of a paladin's damage output \emph{proc subsequent instances of damage}, and also \emph{have internal cooldowns}.
This means that instances of a paladin's damage output are highly \emph{correlated} with one another.

Let us consider the case of the projected damage for a simple naked swing in the presence of a Windfury Totem.
The chance for the first attack to glance or crit is independent from the Windfury attack, but if the first attack is \emph{dodged}, then any possible Windfury attack is also fully negated.
In this way, dodge \emph{correlates} the two attacks, meaning that we cannot naively combine instances of $d$ when considering the sequence of damage because the probability for a dodge is encoded into $d$.
It is therefore advantageous to formulate an expression for the projected physical damage of an attack \emph{independently of the chance for it to be dodged}.

This is complicated by \pdodge affecting the relative magnitudes of \phit, \pglance, and \pcrit arising from the use of a single-roll attack table.
Therefore even when deriving an expression that describes only the case when the attack is \emph{not dodged}, we still expect to see \pdodge as a relevant factor.

We therefore define \dphys as the \emph{average damage from an attack that connects with the enemy}, i.e. is not dodged.
\begin{equation}
	\dphys = \dave \frac{(\pglance \fglance + \pcrit \fcrit + (1 - \pglance - \pdodge - \pcrit)}{1 - \pdodge}
\end{equation}

Note that despite the presence of two \pdodge terms, the average damage is projected \emph{only onto scenarios in which the attack connects and is not negated}.
We also see the well-established concept of a maximum limit on critical hit probability, or ``crit cap'' manifest in this equation, in the form of the necessary inequality:
\begin{equation}
	\pcrit \leq (1-\pglance - \pdodge)
\end{equation}
As glance and dodge have higher \emph{precedence} than crit, it can never push them off the attack table.
As such, any crit probability above $1 - \pglance - \pdodge$ will be ignored.
This concept is more significant (and more usually encountered) when considering dual-wielding classes and miss chance, but in our example we are assuming a hit-capped ret paladin with a two-handed weapon and can neglect misses.

\subsubsection{An Example Calculation with Windfury Attack}
To see why this formulation is useful, let us now derive the projected damage in the case of a naked windfury swing in these terms.
The typical values for the above for a sensibly-geared ret paladin hitting a Boss level enemy correspond to $\pglance = 0.24$, $\fglance = 0.75$, $\fcrit = 2.06$, with a probability of proccing Windfury $\pwf = 0.2$.

We must also consider that the windfury attack has an amount of bonus attack power, that depends on the rank of the Windfury Totem and the Shaman's talents.
If the ret paladin's attack power is $x$ and the bonus attack power on the Windfury Attack is $y$, the $\dave(\mathrm{wf})$ on the Windfury attack can be expressed as a fraction of the autoattack's \dave as:
\begin{equation*}
	\begin{aligned}
		\frac{\dave(\mathrm{wf})}{\dave} &= \frac{\fimpsanc \ftwoh \fcrusade \left( \wdamave + \frac{(x+y)\ws}{14} \right)}{\fimpsanc \ftwoh \fcrusade \left( \bar{\wdam} + \frac{x\ws}{14} \right)} \\
		&= \frac{\wdamave + \frac{\ws x + \ws y}{14}}{\wdamave + \frac{\ws x}{14}}
	\end{aligned}
\end{equation*}
We can therefore define a scale factor
\begin{equation}
	\fwf = \frac{\wdamave + \frac{\ws x + \ws y}{14}}{\wdamave + \frac{\ws x}{14}}
\end{equation}
which is the relative damage of a Windfury hit (with bonus Windfury attack power $y$, a paladin's attack power $x$, and a weapon speed \ws) to a normal autoattack hit.

The projected damage of a naked hit with Windfury active and a non-zero \pdodge is then:
\begin{equation*}
	\begin{aligned}
		d_\mathrm{wf} &= (1 - \pdodge) \dphys + (1 - \pdodge) (1 - \pdodge) \pwf \fwf \dphys\\
		&= (1 - \pdodge) \dphys + (1 - 2\pdodge + \pdodge^2) \pwf \fwf \dphys \\
		&= \dphys(1 - \pdodge + \pwf\fwf(1 - 2\pdodge + \pdodge^2))
	\end{aligned}
\end{equation*}
In the limit that the paladin's attack power $x$ is very high, such the bonus attack power $y$ on their windfury attack becomes negligible and \fwf tends to 1, we find
\begin{equation*}
	\begin{aligned}
		d_\mathrm{wf}  &= \dphys(1 - \pdodge + \pwf(1 - 2\pdodge + \pdodge^2))
	\end{aligned}
\end{equation*}
This equation is expressed entirely in terms of our \dphys variable, the dodge chance, and the windfury proc chance.

Substituting in an example \pdodge of $0.065$, for a hypothetical ret paladin with no Expertise, we find $d_\mathrm{wf} = 1.109845~\dphys$, while for a phase two BiS Blood Elf ret paladin with 24 Expertise and therefore a \pdodge of 0.005, we find $d_\mathrm{wf} = 1.193005~\dphys$.

We note that because \pdodge is also included in the expression for \dphys, we cannot naively say that the P2 BiS scenario is better than the no-Expertise scenario by a factor of $1.193005/1.109845 \approx 1.075$.
We can only say how the relative levels of Expertise compare \emph{as a function of Critical Strike Chance}.




%	Because autoattack and seal damage are subject to different random outcomes, it is useful to define separate quantities for average physical attack damage and average holy attack damage.
%	
%	We then define the average physical attack damage in terms of \dave, accounting for glancing blows and crit chance:
%	\begin{equation}
%		\dphys = \dave (\pglance \dglance + \pcrit \dcrit + (1 - \pglance - \pcrit - \pdodge))
%	\end{equation}
%	where \pglance and \dglance are the probability and damage scaling of a glancing blow, \pcrit and \dcrit are the probability and damage scaling of a critical hit, and \pdodge is the chance for the attack to be dodged.
%	Note that the above does not account for the chance for the attack to be dodged, but includes \pdodge as it affects the relative weightings of normal hits, crits, and glancing blows when the dodge outcome is not totally removed from the attack table.
%	
%	\todo{the above is incomplete with respect to the dodge chance, we are missing probability in the equation. need to better explain the expansion of regular hit to consume the rest of the attack table, and dodge affecting this and therefore the relative probabilities of crits, glances, and hits.}
%	
%


	\appendix
	
\end{document}
